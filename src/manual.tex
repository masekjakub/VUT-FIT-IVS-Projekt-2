\documentclass{article}

% Language setting
% Replace `english' with e.g. `spanish' to change the document language
\usepackage[english]{babel}

% Set page size and margins
% Replace `letterpaper' with `a4paper' for UK/EU standard size
\usepackage[letterpaper,top=2cm,bottom=2cm,left=3cm,right=3cm,marginparwidth=1.75cm]{geometry}

% Useful packages
\usepackage{amsmath}
\usepackage{float}
\usepackage{graphicx}
\usepackage[colorlinks=true, allcolors=blue]{hyperref}

\title{IVS project 2\\User manual}
\author{EXPECT\_EQ(oznuk, false)}

\begin{document}
\maketitle

%\begin{abstract}
%Your abstract.
%\end{abstract}

\section{Introduction}

Your introduction goes here! Simply start writing your document and use the Recompile button to view the updated PDF preview. Examples of commonly used commands and features are listed below, to help you get started.

Once you're familiar with the editor, you can find various project setting in the Overleaf menu, accessed via the button in the very top left of the editor. To view tutorials, user guides, and further documentation, please visit our \href{https://www.overleaf.com/learn}{help library}, or head to our plans page to \href{https://www.overleaf.com/user/subscription/plans}{choose your plan}.


\section{Installation and uninstallation}
Before you can start with installation you need to make sure you run Windows 10 on your device and that you have the proper user rights to install new programs on your device. 

\subsection{Installation}
To successfully install your calculator program you need to follow these steps.

\begin{enumerate}
\item Click on the [this] button.
  \begin{figure}[H]
    \centering
    \includegraphics[width = .3\linewidth]{example-image-a}
    \caption{A figure caption}
  \end{figure}
  
\item Click on the [this] button.
  \begin{figure}[H]
    \centering
    \includegraphics[width = .3\linewidth]{example-image-a}
    \caption{A figure     nnknkncaption}
  \end{figure}

\item Click on the [this] button.
\item Click on the [this] button.


\end{enumerate}
Installation Installation Installation Installation Installation Installation Installation Installation Installation Installation Installation Installation Installation Installation Installation Installation.


\subsection{Uninstallation}

\begin{enumerate}
\item Click on the [this] button.
  \begin{figure}[H]
    \centering
    \includegraphics[width = .3\linewidth]{example-image-a}
    \caption{A figure caption}
  \end{figure}
  
\item Click on the [this] button.
  \begin{figure}[H]
    \centering
    \includegraphics[width = .3\linewidth]{example-image-a}
    \caption{A figure }
  \end{figure}

\item Click on the [this] button.
\item Click on the [this] button.
\end{enumerate}


\section{Calculator view}
After successfully running the executable the main window shows up. The window contains of function buttons and display panel. Buttons are used for entering expressions and can be replaced by physical keyboard. Display panel returns results of calculations and shows the history of calculations.
  \begin{figure}[H]
    \centering
    \includegraphics[width = .3\linewidth]{example-image-a}
  \caption{Calculator view}
\end{figure}

\begin{enumerate}
\item Display panel - Displaying entry and history.
\item Number buttons - Used for entering numbers.
\item Result button - Returns result of current expression.
\item Backspace button - Deletes last added character.
\item C button - Clears an entry.
\item Mathematical functions - Enters mathematical function.
\item $\pi$ Button - Enters number $\pi$.
\end{enumerate}

\subsection{Mathematical functions}
\label{mf}
Calculator can compute various mathematical functions. Different function have different number of operands. In case of every 2 operand function (\emph{+, -, *, /, n\^{}x, n$\sqrt{x}$}) you enter operands in \emph{operand1 function operand2} order. In case of factorial (\emph{x!}) function you enter operand in \emph{operand function} order. In case of natural logarithm you enter operand in \emph{function operand} order.
\\For example the $\frac{\sqrt{4}}{3-ln2}$ expression would be written in this order:
\begin{enumerate}
\item 2 number button
\item Root button
\item 4 number button
\item Division button
\item 3 number button
\item Subtraction button
\item Natural logarithm button
\item 2 number button
\end{enumerate}
\\
Here you can see overview of the mathematical functions:
\begin{center}
\begin{tabular}{| l | c | c | c |}
\hline
Function & Symbol & Number of operands & Input Order\\\hline\hline
Addition & + & 2 & \emph{operand1 function operand2}\\\hline
Subtraction & - & 2 & \emph{operand1 function operand2}\\\hline
Multiplication & * & 2 & \emph{operand1 function operand2}\\\hline
Division & / & 2 & \emph{operand1 function operand2}\\\hline
Power & n\^{}x & 2 & \emph{operand1 function operand2}\\\hline
Root & n$\sqrt{x}$  & 2 & \emph{operand1 function operand2}\\\hline
Factorial & x! & 1 & \emph{operand function}\\\hline
Natural logarithm & ln x & 1 & \emph{function operand}\\\hline
\end{tabular}
\end{center}

\subsection{How to add Comments and Track Changes}

Comments can be added to your project by highlighting some text and clicking ``Add comment'' in the top right of the editor pane. To view existing comments, click on the Review menu in the toolbar above. To reply to a comment, click on the Reply button in the lower right corner of the comment. You can close the Review pane by clicking its name on the toolbar when you're done reviewing for the time being.

Track changes are available on all our \href{https://www.overleaf.com/user/subscription/plans}{premium plans}, and can be toggled on or off using the option at the top of the Review pane. Track changes allow you to keep track of every change made to the document, along with the person making the change. 

\subsection{How to add Lists}

You can make lists with automatic numbering \dots

\begin{enumerate}
\item Like this,
\item and like this.
\end{enumerate}
\dots or bullet points \dots
\begin{itemize}
\item Like this,
\item and like this.
\end{itemize}

\subsection{How to write Mathematics}

\LaTeX{} is great at typesetting mathematics. Let $X_1, X_2, \ldots, X_n$ be a sequence of independent and identically distributed random variables with $\text{E}[X_i] = \mu$ and $\text{Var}[X_i] = \sigma^2 < \infty$, and let
\[S_n = \frac{X_1 + X_2 + \cdots + X_n}{n}
      = \frac{1}{n}\sum_{i}^{n} X_i\]
denote their mean. Then as $n$ approaches infinity, the random variables $\sqrt{n}(S_n - \mu)$ converge in distribution to a normal $\mathcal{N}(0, \sigma^2)$.


\subsection{How to change the margins and paper size}

Usually the template you're using will have the page margins and paper size set correctly for that use-case. For example, if you're using a journal article template provided by the journal publisher, that template will be formatted according to their requirements. In these cases, it's best not to alter the margins directly.

If however you're using a more general template, such as this one, and would like to alter the margins, a common way to do so is via the geometry package. You can find the geometry package loaded in the preamble at the top of this example file, and if you'd like to learn more about how to adjust the settings, please visit this help article on \href{https://www.overleaf.com/learn/latex/page_size_and_margins}{page size and margins}.

\subsection{How to change the document language and spell check settings}

Overleaf supports many different languages, including multiple different languages within one document. 

To configure the document language, simply edit the option provided to the babel package in the preamble at the top of this example project. To learn more about the different options, please visit this help article on \href{https://www.overleaf.com/learn/latex/International_language_support}{international language support}.

To change the spell check language, simply open the Overleaf menu at the top left of the editor window, scroll down to the spell check setting, and adjust accordingly.

\subsection{How to add Citations and a References List}

You can simply upload a \verb|.bib| file containing your BibTeX entries, created with a tool such as JabRef. You can then cite entries from it, like this: \cite{greenwade93}. Just remember to specify a bibliography style, as well as the filename of the \verb|.bib|. You can find a \href{https://www.overleaf.com/help/97-how-to-include-a-bibliography-using-bibtex}{video tutorial here} to learn more about BibTeX.

If you have an \href{https://www.overleaf.com/user/subscription/plans}{upgraded account}, you can also import your Mendeley or Zotero library directly as a \verb|.bib| file, via the upload menu in the file-tree.

\subsection{Good luck!}

We hope you find Overleaf useful, and do take a look at our \href{https://www.overleaf.com/learn}{help library} for more tutorials and user guides! Please also let us know if you have any feedback using the Contact Us link at the bottom of the Overleaf menu --- or use the contact form at \url{https://www.overleaf.com/contact}.

\bibliographystyle{alpha}
\bibliography{sample}

\end{document}
